%%
%% Automatically generated file from DocOnce source
%% (https://github.com/doconce/doconce/)
%% doconce format latex execute.do.txt --execute
%%


%-------------------- begin preamble ----------------------

\documentclass[%
oneside,                 % oneside: electronic viewing, twoside: printing
final,                   % draft: marks overfull hboxes, figures with paths
chapterprefix=true,      % "Chapter" word at beginning of each chapter
open=right,              % start new chapters on odd-numbered pages
10pt]{book}

\listfiles               %  print all files needed to compile this document

\usepackage{relsize,epsfig,makeidx,color,setspace,amsmath,amsfonts,amssymb}
\usepackage[table]{xcolor}
\usepackage{bm,ltablex,microtype}

\usepackage{graphicx}

\usepackage{fancyvrb} % packages needed for verbatim environments
\usepackage{fancyvrb}

\usepackage[T1]{fontenc}
%\usepackage[latin1]{inputenc}
\usepackage{ucs}
\usepackage[utf8x]{inputenc}

\usepackage{lmodern}         % Latin Modern fonts derived from Computer Modern

% Hyperlinks in PDF:
\definecolor{linkcolor}{rgb}{0,0,0.4}
\usepackage{hyperref}
\hypersetup{
    breaklinks=true,
    colorlinks=true,
    linkcolor=linkcolor,
    urlcolor=linkcolor,
    citecolor=black,
    filecolor=black,
    %filecolor=blue,
    pdfmenubar=true,
    pdftoolbar=true,
    bookmarksdepth=3   % Uncomment (and tweak) for PDF bookmarks with more levels than the TOC
    }
%\hyperbaseurl{}   % hyperlinks are relative to this root

\setcounter{tocdepth}{3}  % levels in table of contents

% prevent orhpans and widows
\clubpenalty = 10000
\widowpenalty = 10000

% Redefine double page clear to make it a blank page without headers
% (from BYUTextbook)
\makeatletter
\def\cleardoublepage{\clearpage\if@twoside \ifodd\c@page\else
\hbox{}
\thispagestyle{empty}
\newpage
\if@twocolumn\hbox{}\newpage\fi\fi\fi}
\makeatother
% These commands fiddle with the space left for page numbers in the TOC
% (from BYUTextbook)
\makeatletter
%\renewcommand{\@pnumwidth}{2em}
%\renewcommand{\@tocrmarg}{2.85em}
\makeatother

% Make sure blank even-numbered pages before new chapters are
% totally blank with no header
\newcommand{\clearemptydoublepage}{\clearpage{\pagestyle{empty}\cleardoublepage}}
%\let\cleardoublepage\clearemptydoublepage % caused error in the toc

% --- end of standard preamble for documents ---


% insert custom LaTeX commands...

\raggedbottom
\makeindex
\usepackage[totoc]{idxlayout}   % for index in the toc
\usepackage[nottoc]{tocbibind}  % for references/bibliography in the toc

%-------------------- end preamble ----------------------

\begin{document}

% matching end for #ifdef PREAMBLE

\newcommand{\exercisesection}[1]{\subsection*{#1}}

\input{newcommands_bfmath}
\input{newcommands_replace}

% ------------------- main content ----------------------

\chapter{Automatic execution of code blocks}

Convert this document to \texttt{ipynb}, \texttt{latex} or \texttt{html} with e.g.:



\begin{Verbatim}[numbers=none,fontsize=\fontsize{9pt}{9pt},baselinestretch=0.95]
doconce format ipynb execute.do.txt --execute

\end{Verbatim}


\section{Code blocks in different languages}
\subsection{Python}

Python code




\begin{Verbatim}[numbers=none,fontsize=\fontsize{9pt}{9pt},baselinestretch=0.95]
for i in [1,2,3]:
  print(i)

\end{Verbatim}

\begin{Verbatim}[numbers=none,fontsize=\fontsize{9pt}{9pt},baselinestretch=0.95]



\end{Verbatim}



\begin{Verbatim}[numbers=none,fontsize=\fontsize{9pt}{9pt},baselinestretch=0.95]
print(i)

\end{Verbatim}

\begin{Verbatim}[numbers=none,fontsize=\fontsize{9pt}{9pt},baselinestretch=0.95]

\end{Verbatim}

\subsection{Bash}
Bash code 




\begin{Verbatim}[numbers=none,fontsize=\fontsize{9pt}{9pt},baselinestretch=0.95]
if [ 1 -eq 1 ] ; then echo 1; fi
var_bash=10

\end{Verbatim}

\begin{Verbatim}[numbers=none,fontsize=\fontsize{9pt}{9pt},baselinestretch=0.95]

\end{Verbatim}



\begin{Verbatim}[numbers=none,fontsize=\fontsize{9pt}{9pt},baselinestretch=0.95]
echo $var_bash

\end{Verbatim}

\begin{Verbatim}[numbers=none,fontsize=\fontsize{9pt}{9pt},baselinestretch=0.95]

\end{Verbatim}

\subsection{Julia}

Julia code




\begin{Verbatim}[numbers=none,fontsize=\fontsize{9pt}{9pt},baselinestretch=0.95]
var_julia = 11
print(var_julia)

\end{Verbatim}

\begin{Verbatim}[numbers=none,fontsize=\fontsize{9pt}{9pt},baselinestretch=0.95]
11\epy






\begin{Verbatim}[numbers=none,fontsize=\fontsize{9pt}{9pt},baselinestretch=0.95]
for n = 2:4
  var_julia = var_julia + n
end
print(var_julia)

\end{Verbatim}

\begin{Verbatim}[numbers=none,fontsize=\fontsize{9pt}{9pt},baselinestretch=0.95]
20\epy

\subsection{R }

R code







\begin{Verbatim}[numbers=none,fontsize=\fontsize{9pt}{9pt},baselinestretch=0.95]
x <- 1:3
print(x)
#pdf("plot.pdf")
plot(x)
#dev.off()

\end{Verbatim}

\begin{Verbatim}[numbers=none,fontsize=\fontsize{9pt}{9pt},baselinestretch=0.95]
[1] 1 2 3
\end{Verbatim}
\begin{center}
   \includegraphics[width=0.8\textwidth]{.doconce_figure_cache/XXX.png}
\end{center}


\subsection{Other languages}

Then Cython (with -h option so it is hidden in html/sphinx):




\begin{Verbatim}[numbers=none,fontsize=\fontsize{9pt}{9pt},baselinestretch=0.95]
cpdef f(double x):
    return x + 1

\end{Verbatim}


Java code




\begin{Verbatim}[numbers=none,fontsize=\fontsize{9pt}{9pt},baselinestretch=0.95]
for (int i = 0; i < 5; i++) {
  System.out.println(i);
}

\end{Verbatim}


Javascript code


\begin{Verbatim}[numbers=none,fontsize=\fontsize{9pt}{9pt},baselinestretch=0.95]
for (var x in [0,1,2]) {console.log(x)}

\end{Verbatim}


matlab code




\begin{Verbatim}[numbers=none,fontsize=\fontsize{9pt}{9pt},baselinestretch=0.95]
for i = 1:2:10
  disp(A(i))
end

\end{Verbatim}


html code


\begin{Verbatim}[numbers=none,fontsize=\fontsize{9pt}{9pt},baselinestretch=0.95]
<a href='test'></a>

\end{Verbatim}


C code












\begin{Verbatim}[numbers=none,fontsize=\fontsize{9pt}{9pt},baselinestretch=0.95]
#include <stdio.h>

int main() {
  int i;

  for (i = 1; i < 11; ++i)
  {
    printf("%d ", i);
  }
  return 0;
}

\end{Verbatim}


\section{Code block environments}

Hidden execution cells (\texttt{pyhid}, \texttt{pycod-e}) can be used to perform operations (e.g.~imports, variable initializations) without showing any cell.  
The \texttt{pyhid} environment executes and hides the cell in formats other than .ipynb:

This is a normal python block using the \texttt{pycod} environment


\begin{Verbatim}[numbers=none,fontsize=\fontsize{9pt}{9pt},baselinestretch=0.95]
print('pycod')

\end{Verbatim}

\begin{Verbatim}[numbers=none,fontsize=\fontsize{9pt}{9pt},baselinestretch=0.95]
pycod
\end{Verbatim}







The \texttt{pycod-e} environment executes but hides the cell also in .ipynb files:







\texttt{pycod} is a normal cell that should execute automatically when using \texttt{--execute}. Note that this cells relies on code executed in a previous hidden cell:






\begin{Verbatim}[numbers=none,fontsize=\fontsize{9pt}{9pt},baselinestretch=0.95]
print(sys.version)
b = 2
c = a + b
print("The result is {}".format(c))
c

\end{Verbatim}

\begin{Verbatim}[numbers=none,fontsize=\fontsize{9pt}{9pt},baselinestretch=0.95]
3.8.0 (default, Feb 25 2021, 22:10:10) 
[GCC 8.4.0]
The result is 3
\end{Verbatim}
\begin{Verbatim}[numbers=none,fontsize=\fontsize{9pt}{9pt},baselinestretch=0.95]
3\epy

% !split
\section{Special environments}

The \texttt{*-t} environment (e.g. \texttt{pycod-t}) formats a cell to text, and can be used to print an example




\begin{Verbatim}[numbers=none,fontsize=\fontsize{9pt}{9pt},baselinestretch=0.95]
# This is a for-loop example
for i in [0,10]:
  print(i)

\end{Verbatim}


The \texttt{*out}  (e.g. \texttt{pycod-out}) environment can be used to write a cell output:



\begin{Verbatim}[numbers=none,fontsize=\fontsize{9pt}{9pt},baselinestretch=0.95]
# This is a text cell using pycod-t
1/0

\end{Verbatim}




\begin{Verbatim}[numbers=none,fontsize=\fontsize{9pt}{9pt},baselinestretch=0.95]
# This is a output cell using the `pycod-out` environment
1/0: You cannot divide by zero

\end{Verbatim}

\begin{Verbatim}[numbers=none,fontsize=\fontsize{9pt}{9pt},baselinestretch=0.95]
# This is a output cell using the `pycod-out` environment
1/0: You cannot divide by zero
\end{Verbatim}

The \texttt{-h} postfix can be used in the \texttt{html} format to show a Show/Hide button that toggles the code visibility. 

The \texttt{pyscpro} environment creates an interactive cell using \href{{https://github.com/sagemath/sagecell/}}{Sage} in the \texttt{html} format

% !split
\section{Plotting}

This is a cell that should plot and output:






\begin{Verbatim}[numbers=none,fontsize=\fontsize{9pt}{9pt},baselinestretch=0.95]
from pylab import *
x = linspace(0, 10, 100)
plot(x, x*x)
show()

\end{Verbatim}

\begin{center}
   \includegraphics[width=0.8\textwidth]{.doconce_figure_cache/XXX.pdf}
\end{center}


To improve quality when exporting to {\LaTeX}, the following code has automatically
been run to enable PDF export in notebooks.




\begin{Verbatim}[numbers=none,fontsize=\fontsize{9pt}{9pt},baselinestretch=0.95]
from IPython.display import set_matplotlib_formats
set_matplotlib_formats('png', 'pdf')

\end{Verbatim}


% !split
\section{Ignore output}

Predefined output can be omitted by passing \Verb!--ignore_output! to DocOnce.
This will remove all environments ending with \texttt{out}.




\begin{Verbatim}[numbers=none,fontsize=\fontsize{9pt}{9pt},baselinestretch=0.95]
a = 2
print(a)

\end{Verbatim}

\begin{Verbatim}[numbers=none,fontsize=\fontsize{9pt}{9pt},baselinestretch=0.95]

\end{Verbatim}



\begin{Verbatim}[numbers=none,fontsize=\fontsize{9pt}{9pt},baselinestretch=0.95]


\end{Verbatim}

\begin{Verbatim}[numbers=none,fontsize=\fontsize{9pt}{9pt},baselinestretch=0.95]

\end{Verbatim}

% !split
\section{Code with errors}

If code contains errors, it will still be run and the exception shown as part
of the output:




\begin{Verbatim}[numbers=none,fontsize=\fontsize{9pt}{9pt},baselinestretch=0.95]
for a in range(10)
    print(a)

\end{Verbatim}

\begin{Verbatim}[numbers=none,fontsize=\fontsize{9pt}{9pt},baselinestretch=0.95]
  File "<ipython-input-10-b3f86d48875f>", line XXX
    for a in range(10)
                      ^
SyntaxError: invalid syntax
\end{Verbatim}

% !split
\section{Opening files}

The working directory is the same as the .do.txt file.
You may want to use \texttt{os.chdir} to change the directory.




\begin{Verbatim}[numbers=none,fontsize=\fontsize{9pt}{9pt},baselinestretch=0.95]
with open("../LICENSE") as f:
    print(f.read())

\end{Verbatim}

\begin{Verbatim}[numbers=none,fontsize=\fontsize{9pt}{9pt},baselinestretch=0.95]
Copyright (c) 2007-2XXX, Hans Petter Langtangen <hpl@simula.no> and
Simula Resarch Laboratory.

All rights reserved.

Redistribution and use in source and binary forms, with or without
modification, are permitted provided that the following conditions are
met:

    * Redistributions of source code must retain the above copyright
      notice, this list of conditions and the following disclaimer.

    * Redistributions in binary form must reproduce the above copyright
      notice, this list of conditions and the following disclaimer in
      the documentation and/or other materials provided with the
      distribution.

    * Neither the name of Simula Research Laboratory nor the names of
      its contributors may be used to endorse or promote products
      derived from this software without specific prior written
      permission.

THIS SOFTWARE IS PROVIDED BY THE COPYRIGHT HOLDERS AND CONTRIBUTORS
"AS IS" AND ANY EXPRESS OR IMPLIED WARRANTIES, INCLUDING, BUT NOT
LIMITED TO, THE IMPLIED WARRANTIES OF MERCHANTABILITY AND FITNESS FOR
A PARTICULAR PURPOSE ARE DISCLAIMED. IN NO EVENT SHALL THE COPYRIGHT
OWNER OR CONTRIBUTORS BE LIABLE FOR ANY DIRECT, INDIRECT, INCIDENTAL,
SPECIAL, EXEMPLARY, OR CONSEQUENTIAL DAMAGES (INCLUDING, BUT NOT LIMITED
TO, PROCUREMENT OF SUBSTITUTE GOODS OR SERVICES; LOSS OF USE, DATA, OR
PROFITS; OR BUSINESS INTERRUPTION) HOWEVER CAUSED AND ON ANY THEORY OF
LIABILITY, WHETHER IN CONTRACT, STRICT LIABILITY, OR TORT (INCLUDING
NEGLIGENCE OR OTHERWISE) ARISING IN ANY WAY OUT OF THE USE OF THIS
SOFTWARE, EVEN IF ADVISED OF THE POSSIBILITY OF SUCH DAMAGE.

Remarks:

The figure and movie files in doc/manual/* were made by the Doconce
author and is released under the same conditions as Doconce.


\end{Verbatim}

% ------------------- end of main content ---------------

\end{document}

