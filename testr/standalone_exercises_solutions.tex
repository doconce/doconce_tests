








\documentclass[
oneside,                 
final,                   
10pt]{article}

\listfiles               

\usepackage{relsize,makeidx,color,setspace,amsmath,amsfonts,amssymb}
\usepackage[table]{xcolor}
\usepackage{bm,ltablex,microtype}

\usepackage[pdftex]{graphicx}


\usepackage{fancyvrb,framed,moreverb}


\definecolor{orange}{cmyk}{0,0.4,0.8,0.2}
\definecolor{tucorange}{rgb}{1.0,0.64,0}
\definecolor{darkorange}{rgb}{.71,0.21,0.01}
\definecolor{darkgreen}{rgb}{.12,.54,.11}
\definecolor{myteal}{rgb}{.26, .44, .56}
\definecolor{gray}{gray}{0.45}
\definecolor{mediumgray}{gray}{.8}
\definecolor{lightgray}{gray}{.95}
\definecolor{brown}{rgb}{0.54,0.27,0.07}
\definecolor{purple}{rgb}{0.5,0.0,0.5}
\definecolor{darkgray}{gray}{0.25}
\definecolor{darkblue}{rgb}{0,0.08,0.45}
\definecolor{darkblue2}{rgb}{0,0,0.8}
\definecolor{lightred}{rgb}{1.0,0.39,0.28}
\definecolor{lightgreen}{rgb}{0.48,0.99,0.0}
\definecolor{lightblue}{rgb}{0.53,0.81,0.92}
\definecolor{lightblue2}{rgb}{0.3,0.3,1.0}
\definecolor{lightpurple}{rgb}{0.87,0.63,0.87}
\definecolor{lightcyan}{rgb}{0.5,1.0,0.83}

\colorlet{comment_green}{green!50!black}
\colorlet{string_red}{red!60!black}
\colorlet{keyword_pink}{magenta!70!black}
\colorlet{indendifier_green}{green!70!white}


\definecolor{cbg_gray}{rgb}{.95, .95, .95}
\definecolor{bar_gray}{rgb}{.92, .92, .92}

\definecolor{cbg_yellowgray}{rgb}{.95, .95, .85}
\definecolor{bar_yellowgray}{rgb}{.95, .95, .65}

\colorlet{cbg_yellow2}{yellow!10}
\colorlet{bar_yellow2}{yellow!20}

\definecolor{cbg_yellow1}{rgb}{.98, .98, 0.8}
\definecolor{bar_yellow1}{rgb}{.98, .98, 0.4}

\definecolor{cbg_red1}{rgb}{1, 0.85, 0.85}
\definecolor{bar_red1}{rgb}{1, 0.75, 0.85}

\definecolor{cbg_blue1}{rgb}{0.87843, 0.95686, 1.0}
\definecolor{bar_blue1}{rgb}{0.7,     0.95686, 1}


\usepackage[T1]{fontenc}

\usepackage{ucs}
\usepackage[utf8x]{inputenc}

\usepackage{lmodern}         


\definecolor{linkcolor}{rgb}{0,0,0.4}
\usepackage{hyperref}
\hypersetup{
    breaklinks=true,
    colorlinks=true,
    linkcolor=linkcolor,
    urlcolor=linkcolor,
    citecolor=black,
    filecolor=black,
    
    pdfmenubar=true,
    pdftoolbar=true,
    bookmarksdepth=3   
    }


\setcounter{tocdepth}{2}  


\clubpenalty = 10000
\widowpenalty = 10000






\raggedbottom
\makeindex
\usepackage[totoc]{idxlayout}   
\usepackage[nottoc]{tocbibind}  



\begin{document}




\newcommand{\exercisesection}[1]{\subsection*{#1}}




\exercisesection{Example \thedoconceexercisecounter: Examples can be typeset as exercises}
                             

\paragraph{a) Solution.}
The answer to this subproblem can be written here.

\paragraph{b) Solution.}
The answer to this other subproblem goes here,
maybe over multiple doconce input lines.

\exercisesection{Problem \thedoconceexercisecounter: Flip a Coin}
                             


\paragraph{Answer.}
If the \texttt{random.random()} function returns a number $<1/2$, let it be
head, otherwise tail. Repeat this $N$ number of times.

\paragraph{a) Solution.}









\begin{Verbatim}[numbers=none,fontsize=\fontsize{9pt}{9pt},baselinestretch=0.95,xleftmargin=2mm]
import sys, random
N = int(sys.argv[1])
heads = 0
for i in range(N):
    r = random.random()
    if r <= 0.5:
        heads += 1
print('Flipping a coin 

\end{Verbatim}


\paragraph{Answer.}
\texttt{np.sum(np.where(r <= 0.5, 1, 0))} or \texttt{np.sum(r <= 0.5)}.

\exercisesection{Project \thedoconceexercisecounter: Explore Distributions of Random Circles}
                             


\paragraph{Answer.}
Here goes the short answer to part a).

\paragraph{a) Solution.}
Here goes a full solution to part a).

\exercisesection{Exercise \thedoconceexercisecounter: Determine some Distance}
                             


\paragraph{Solution.}
Here goes a full solution of the whole exercise.
With some math $a=b$ in this solution:
\[ \hbox{math in solution: } a = b \]
And code \texttt{a=b} in this solution:


\begin{Verbatim}[numbers=none,fontsize=\fontsize{9pt}{9pt},baselinestretch=0.95,xleftmargin=2mm]
a = b  # code in solution

\end{Verbatim}

End of solution is here.

\paragraph{Answer.}
Short answer to subexercise a).
With math in answer: $a=b$.

\paragraph{b) Solution.}
Here goes the solution of this subexercise.

\exercisesection{Example \thedoconceexercisecounter: Just an example}
                             





\end{document}


