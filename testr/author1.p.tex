%%
%% File automatically generated using DocOnce (https://github.com/doconce/doconce/):
%% doconce format latex author1.do.txt 
%%
% #ifdef PTEX2TEX_EXPLANATION
%%
%% The file follows the ptex2tex extended LaTeX format, see
%% ptex2tex: https://code.google.com/p/ptex2tex/
%%
%% Run
%%      ptex2tex myfile
%% or
%%      doconce ptex2tex myfile
%%
%% to turn myfile.p.tex into an ordinary LaTeX file myfile.tex.
%% (The ptex2tex program: https://code.google.com/p/ptex2tex)
%% Many preprocess options can be added to ptex2tex or doconce ptex2tex
%%
%%      ptex2tex -DMINTED myfile
%%      doconce ptex2tex myfile envir=minted
%%
%% ptex2tex will typeset code environments according to a global or local
%% .ptex2tex.cfg configure file. doconce ptex2tex will typeset code
%% according to options on the command line (just type doconce ptex2tex to
%% see examples). If doconce ptex2tex has envir=minted, it enables the
%% minted style without needing -DMINTED.
% #endif

% #define PREAMBLE

% #ifdef PREAMBLE
%-------------------- begin preamble ----------------------

\documentclass[%
oneside,                 % oneside: electronic viewing, twoside: printing
final,                   % draft: marks overfull hboxes, figures with paths
chapterprefix=true,      % "Chapter" word at beginning of each chapter
open=right,              % start new chapters on odd-numbered pages
10pt]{book}

\listfiles               %  print all files needed to compile this document

\usepackage{relsize,epsfig,makeidx,color,setspace,amsmath,amsfonts,amssymb}
\usepackage[table]{xcolor}
\usepackage{bm,ltablex,microtype}

\usepackage{graphicx}

\usepackage{ptex2tex}
% #ifdef MINTED
\usepackage{minted}
\usemintedstyle{default}
% #endif
\usepackage{fancyvrb}

\usepackage[T1]{fontenc}
%\usepackage[latin1]{inputenc}
\usepackage{ucs}
\usepackage[utf8x]{inputenc}

\usepackage{lmodern}         % Latin Modern fonts derived from Computer Modern

% Hyperlinks in PDF:
\definecolor{linkcolor}{rgb}{0,0,0.4}
\usepackage{hyperref}
\hypersetup{
    breaklinks=true,
    colorlinks=true,
    linkcolor=linkcolor,
    urlcolor=linkcolor,
    citecolor=black,
    filecolor=black,
    %filecolor=blue,
    pdfmenubar=true,
    pdftoolbar=true,
    bookmarksdepth=3   % Uncomment (and tweak) for PDF bookmarks with more levels than the TOC
    }
%\hyperbaseurl{}   % hyperlinks are relative to this root

\setcounter{tocdepth}{3}  % levels in table of contents

% prevent orhpans and widows
\clubpenalty = 10000
\widowpenalty = 10000

% Redefine double page clear to make it a blank page without headers
% (from BYUTextbook)
\makeatletter
\def\cleardoublepage{\clearpage\if@twoside \ifodd\c@page\else
\hbox{}
\thispagestyle{empty}
\newpage
\if@twocolumn\hbox{}\newpage\fi\fi\fi}
\makeatother
% These commands fiddle with the space left for page numbers in the TOC
% (from BYUTextbook)
\makeatletter
%\renewcommand{\@pnumwidth}{2em}
%\renewcommand{\@tocrmarg}{2.85em}
\makeatother

% Make sure blank even-numbered pages before new chapters are
% totally blank with no header
\newcommand{\clearemptydoublepage}{\clearpage{\pagestyle{empty}\cleardoublepage}}
%\let\cleardoublepage\clearemptydoublepage % caused error in the toc

% --- end of standard preamble for documents ---




% References to labels in external documents:
\usepackage{xr}

\externaldocument{testdoc}

% Add external .aux files to \listfiles list:
\makeatletter
\@addtofilelist{testdoc.aux}
\makeatother


% insert custom LaTeX commands...


\raggedbottom
\makeindex
\usepackage[totoc]{idxlayout}   % for index in the toc
\usepackage[nottoc]{tocbibind}  % for references/bibliography in the toc

%-------------------- end preamble ----------------------

\begin{document}

% matching end for #ifdef PREAMBLE
% #endif

\newcommand{\exercisesection}[1]{\subsection*{#1}}

\renewcommand{\u}{\pmb{u}}
\newcommand{\f}{\bm{f}}
\newcommand{\xbm}{\bm{x}}
\newcommand{\normalvecbm}{\bm{n}}
\newcommand{\ubm}{\bm{u}}

\newcommand{\x}{\pmb{x}}
\newcommand{\normalvec}{\pmb{n}}
\newcommand{\Ddt}[1]{\frac{D#1}{dt}}
\newcommand{\halfi}{1/2}
\newcommand{\half}{\frac{1}{2}}
\newcommand{\report}{test report}

% ------------------- main content ----------------------



% ----------------- title -------------------------

\thispagestyle{empty}

\begin{center}
{\LARGE\bf
\begin{spacing}{1.25}
Test of one author at one institution
\end{spacing}
}
\end{center}

% ----------------- author(s) -------------------------

\begin{center}
{\bf John Doe (\texttt{doe@cyberspace.net})}
\end{center}

    \begin{center}
% List of all institutions:
\centerline{{\small Cyberspace Inc.}}
\end{center}
    
% ----------------- end author(s) -------------------------

% --- begin date ---
\begin{center}
Jan 32, 2100
\end{center}
% --- end date ---

\vspace{1cm}


% Externaldocument: testdoc

\chapter{Generalized References}
\label{genrefs}

Sometimes a series of individual documents may be assembled to one
large document. The assembly impacts how references to sections
are written: when referring to a section in the same document, a label
can be used, while references to sections in other documents are
written differently, sometimes involving a link (URL) and a citation.
Especially if both the individual documents and the large assembly document
are to exist side by side, a flexible way of referencing is needed.
For this purpose, DocOnce offers \emph{generalized references} which allows
a reference to have two different formulations, one for internal
references and one for external references. Since {\LaTeX} supports
references to labels in external documents via the \texttt{xr} package,
the generalized references in DocOnce has a syntax that may utilize
the \texttt{xr} feature in {\LaTeX}.

The syntax of generalized references reads


\bdat
ref[internal][cite][external]

\edat

If all standard \texttt{ref} references (with curly braces)
in the text \texttt{internal} are references
to labels in the present document, the above \texttt{ref} command is replaced
by the text \texttt{internal}. Otherwise, if cite is non-empty and the format
is \texttt{latex} or \texttt{pdflatex} one assumes that the references in \texttt{internal}
are to external documents declared by a comment line \Verb!# Externaldocuments: testdoc, mydoc! (usually after the title, authors,
and date). In this case the output text is \texttt{internal cite} and the
{\LaTeX} package \texttt{xr} is used to handle the labels in the external documents.
When referring to a complete chapter (not a section in it), which
corresponds to a complete external document, it does not make sense
to write out \texttt{internal cite} since the \texttt{internal} reference is a
chapter number. In such cases, the \texttt{internal} syntax can be used,
and if the label is in another {\LaTeX} document, the output is just \texttt{cite}.
For all
output formats other than \texttt{latex} and \texttt{pdflatex}, the \texttt{external}
text will be the output.

Here is an example on a specific generalized reference to a section
in a document:









\bdat
As explained in
ref[Section ref{subsec:ex}][in "Langtangen, 2012":
"https://hplgit.github.io/doconce/test/demo_testdoc.html#subsec:ex"
cite{testdoc:12}][a "section":
"https://hplgit.github.io/doconce/test/demo_testdoc.html#subsec:ex" in
the document "A Document for Testing DocOnce":
"https://hplgit.github.io/doconce/test/demo_testdoc.html"
cite{testdoc:12}], DocOnce documents may include tables.

\edat

With \texttt{latex} or \texttt{pdflatex} as output, this translates to



\bdat
As explained in
Section ref{subsec:ex}, DocOnce documents may include tables.

\edat

if the label \Verb!{subsec:ex}! appears in the present DocOnce source, and
otherwise





\bdat
As explained in
Section ref{subsec:ex} in "Langtangen, 2012":
"https://hplgit.github.io/doconce/test/demo_testdoc.html#subsec:ex"
cite{testdoc:12}, DocOnce documents may include tables.

\edat

In a format different from \texttt{latex} and \texttt{pdflatex}, the effective DocOnce
text becomes







\bdat
As explained in
a "section":
"https://hplgit.github.io/doconce/test/demo_testdoc.html#subsec:ex" in
the document "A Document for Testing DocOnce":
"https://hplgit.github.io/doconce/test/demo_testdoc.html"
cite{testdoc:12}, DocOnce documents may include tables.

\edat

The rendered text in the current format \texttt{latex} becomes


\begin{quote}
As explained in
Section~\ref{subsec:ex}in \href{{https://hplgit.github.io/doconce/test/demo_testdoc.html#subsec:ex}}{Langtangen, 2012}
\cite{testdoc:12}, DocOnce documents may include tables.
\end{quote}


A reference to an entire external document, which is usually a chapter
if the reference is internal in the DocOnce source, applies the
\texttt{refch} syntax:









\bdat
As explained in
refch[Chapter ref{ch:testdoc}]["Langtangen, 2012":
"https://hplgit.github.io/doconce/test/demo_testdoc.html"
cite{testdoc:12}][the document
"A Document for Testing DocOnce":
"https://hplgit.github.io/doconce/test/demo_testdoc.html"
cite{testdoc:12}], DocOnce documents may include tables.

\edat

The output now if \texttt{ch:testdoc} is not a label in the document,
becomes in the \texttt{latex} and \texttt{pdflatex} case






\bdat
As explained in
"Langtangen, 2012":
"https://hplgit.github.io/doconce/test/demo_testdoc.html"
cite{testdoc:12}, DocOnce documents may include tables.

\edat

That is, the internal reference \texttt{Chapter ...} is omitted since
it is not meaningful to refer to an external document as "Chapter".
The resulting rendered text in the current format \texttt{latex} becomes


\begin{quote}
As explained in
\href{{https://hplgit.github.io/doconce/test/demo_testdoc.html}}{Langtangen, 2012}
\cite{testdoc:12}, DocOnce documents may include tables.
\end{quote}


Note that {\LaTeX} cannot
have links to local files, so a complete URL on the form
\texttt{https://...} must be used.

And here is another example with internal references only:




\bdat
Generalized references are described in ref[Section ref{genrefs}][dummy1][
dummy2].

\edat

The text is rendered to


\begin{quote}
Generalized references are described in
Section~\ref{genrefs}.
\end{quote}


\chapter{Test of math}

% Here we test the chapter heading to see if latex output then has
% book style rather than article style.

Inline math, $a=b$, is the only math in this document.

% Need BIBFILE because of \cite{} examples
\clearemptydoublepage

\bibliographystyle{plain}
\bibliography{papers}


% ------------------- end of main content ---------------

% #ifdef PREAMBLE
\end{document}
% #endif

